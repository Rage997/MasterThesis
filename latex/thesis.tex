\documentclass[mscthesis]{usiinfthesis}
\usepackage{lipsum}


\usepackage{listings}

\lstdefinelanguage{algebra}
{morekeywords={import,sort,constructors,observers,transformers,axioms,if,
else,end},
sensitive=false,
morecomment=[l]{//s},
}



\title{Alien species modelling via relational event models} %compulsory
%\specialization{Dependable Distributed Systems}%optional
%\subtitle{Subtitle: Reinventing the World} %optional 
\author{Niccol\`o Zuppichini} %compulsory
\begin{committee}
\advisor{Prof.}{Ernst-Jan Camiel}{Wit} %compulsory
\coadvisor{Prof.}{Igor}{Artico}{} %optional
\end{committee}
\Day{15} %compulsory
\Month{June} %compulsory
\Year{2022} %compulsory, put only the year
\place{Lugano} %compulsory

\dedication{To my beloved} %optional
\openepigraph{Someone said \dots}{Someone} %optional

%\makeindex %optional, also comment out \theindex at the end

\begin{document}

\maketitle %generates the titlepage, this is FIXED

\frontmatter %generates the frontmatter, this is FIXED

\begin{abstract}
During the last centuries human research on the interaction of species substantially intensified. However, we know little to nothing about the underlying process that shape the behaviours of alien species invasion across regions, countries and ecosystems. The invasion of species can be understood in terms of actions, or events, which can be properly described as a set of discrete events in which one individual, in our case a species, emits a directed interaction with another, the region or area of invasion. Under this framework, we can describe the invasion of species in time as a temporal bipartite species-region graph. We then present a relation event model (REM) to model the study of co-invasion of species. We then apply this model to a dataset of (TODO size of dataset) ranging from (TODO years dataset). The aim of this paper is hence to study what group of species have the tendency to co-invade a region. 


%TODO should I put reference in introduction?
%TODO should I tak briefly
%TODO introduce briefly computational challenge and HPC

\end{abstract}

%\begin{acknowledgements}
%\lipsum 
%\end{acknowledgements}

%\tableofcontents 
%\listoffigures %optional
%\listoftables %optional

\mainmatter

\chapter{Introduction}
The rate on which alien species invade countries has increased over the last century and it's becoming an issue \citet{intro:rate}



\chapter[Short title]{A chapter title which will run over two lines --- it's for
  testing purpose}

\lipsum[1-2]

\section{The first section}
\lipsum[3-4]

 \section{The second, math section}

\textbf{Theorem 1 (Residue Theorem).}
Let $f$ be analytic in the region $G$ except for the isolated singularities $a_1,a_2,\ldots,a_m$. If $\gamma$ is a closed rectifiable curve in $G$ which does not pass through any of the points $a_k$ and if $\gamma\approx 0$ in $G$ then
\[
\frac{1}{2\pi i}\int_\gamma f = \sum_{k=1}^m n(\gamma;a_k) \text{Res}(f;a_k).
\]
\textbf{Theorem 2 (Maximum Modulus).}
\emph{Let $G$ be a bounded open set in $\mathbb{C}$ and suppose that $f$ is a continuous function on $G^-$ which is analytic in $G$. Then}
\[
\max\{|f(z)|:z\in G^-\}=\max \{|f(z)|:z\in \partial G \}.
\]

\section[third]{A very very long section, titled ``The third section'', with
  a rather  short text alternative (third)}
\lipsum \texttt{Some Test}
\lstset{language=algebra,linewidth=0.95\linewidth,breaklines=true,numbers=left,
basicstyle=\ttfamily,numberstyle=\tiny,escapeinside={//*}{\^^M},
mathescape=true}
\begin{lstlisting}
import IntSpec, ItemSpec;

sort cart; //*\label{sort}

constructors //*\label{begin-sig}
create() $\longrightarrow$ cart;
insert(cart, item) $\longrightarrow$ cart;
observers
amount(cart) $\longrightarrow$ int;
transformers
delete(cart, item) $\longrightarrow$ cart; //*\label{end-sig}

axioms //*\label{begin-axioms}
forall c: cart, i, j: item 

amount(create()) $=$ 0; //*\label{begin-amount}
amount(insert(c,i)) $=$ amount(c) $+$ price(i); //*\label{end-amount}
delete(create(),i) $=$ create(); //*\label{begin-delete}
delete(insert(c,i),j) $=$
if (i =$\:$= j) c
else insert(delete(c,j),i); //*\label{end-axioms}
end
\end{lstlisting}

As you can easily see from the above listing \citet{bbggs:iet07}
define something weird based on the BPEL specification
\citep{bpelspec}.
\nocite{*}

\appendix %optional, use only if you have an appendix

\chapter{Some retarded material}
\section{It's over\dots}
\lipsum 

\backmatter

\chapter{Glossary} %optional

%\bibliographystyle{alpha}
%\bibliographystyle{dcu}
\bibliographystyle{plainnat}
\bibliography{biblio}

%\cleardoublepage
%\theindex %optional, use only if you have an index, must use
	  %\makeindex in the preamble
\lipsum

\end{document}
